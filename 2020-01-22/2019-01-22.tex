%%%%%%%%%%%%%%%%%%%%%%%%%%%%%%%%%%%%%%%%%%%%%%%%%%%%%%%%%%%%%%%%%%%%%%%%%%%
%  ロボティクス研  研究報告用TEXファイル  前刷り (旧田中研フォーマットベース)
%  Resume.tex
%
%  2003.03.14	T.Koyama
%  2005.04.11	H.Ohtake
%  2016.03.24	Y.Higashi
%%%%%%%%%%%%%%%%%%%%%%%%%%%%%%%%%%%%%%%%%%%%%%%%%%%%%%%%%%%%%%%%%%%%%%%%%%%
\documentclass[a4paper]{jarticle}
\usepackage{Resume}
\usepackage[dvipdfmx]{color,graphicx}
\usepackage{slashbox}
\usepackage{amsmath}
\usepackage{textgreek}%ギリシャ文字を立てにするパッケージ
\usepackage{nidanfloat}%横長の図を1ページ内にうまく挿入する

%本文圧縮コマンド(本文参照)
%\renewcommand{\baselinestretch}{0.75}

\begin{document}
\twocolumn[
%%%%%%%%%%%%%%%%%%%%%%%%%%%%%%%%%%%%%%%%%%%%%%%%%%%%%%%%%%%%%%%%%%%%%%%%%%%
%  タイトル・氏名
%%%%%%%%%%%%%%%%%%%%%%%%%%%%%%%%%%%%%%%%%%%%%%%%%%%%%%%%%%%%%%%%%%%%%%%%%%%
\vspace*{10mm}
\begin{center}
	{\Large \gt 2019/01/22 飛翔ロボットミーティング} \\
\end{center}
\begin{flushright}
\begin{tabular}{c@{~}r}
機械設計学専攻	& ロボティクス研究室	\\
18623117		& 中村 翔太		\\
\end{tabular}
\end{flushright}
\vspace{1em}
]

%%%%%%%%%%%%%%%%%%%%%%%%%%%%%%%%%%%%%%%%%%%%%%%%%%%%%%%%%%%%%%%%%%%%%%%%%%%
%  本文
%%%%%%%%%%%%%%%%%%%%%%%%%%%%%%%%%%%%%%%%%%%%%%%%%%%%%%%%%%%%%%%%%%%%%%%%%%%
%%%%%%%%%%%%%%%%%%%%%%%%%%%%%%%%%%%%%%%%%%%%%%%%%%%%%%%%%%%%%%%%%%%%%%%%%%%
%%%%%%%%%%%%%%%%%%%%%%%%%%%%%%%%%%%%%%%%%%%%%%%%%%%%%%%%%%%%%%%%%%%%%%%%%%%


\section{SII学会を終えて}
質問がきた内容を以下に示す.

\begin{itemize}
\itemsep=-1ex
  \item UWBモジュールを用いた位置の推定とモーションキャプチャを用いたシステムとの違いは.
  \item 3点間の移動する飛行制御を行った際,各目標地点にてクアッドロータが揺れる原因は.
  \item 飛行制御と謳っているにも関わらず,高さ方向における制御は行っていないのか.
  \item UWBモジュールはどこの会社が造っているのか.
\end{itemize}

\noindent
と,多くの質問が来た.皆さん(日本人や多くの外国人)が興味を持って聴いてくれたことに嬉しく感じている.これらの質問は修論発表でも質問が予想されるため,修論発表までに質問内容を整理し,的確に答えられるように準備したい.


\section{位置制御結果の真値との比較}
以下に以前行った,オプティカルフローセンサも位置制御に組み込んだ位置制御結果のグラフを示す.またステレオカメラを用いて計測した真値も合わせて示す.

\begin{figure}[htbp]
  \begin{center}
     \includegraphics[width=1\linewidth]{./figure/12_30_20-05-40.png}
     \caption{オプティカルフローセンサも制御に組み込んだ際の位置制御結果(真値との比較)}
     \label{}
  \end{center}
\end{figure}

\noindent
真値には以前にも述べたように5 Hzにてローパスフィルタを掛けている.図からわかるように真値と推定値でズレが有る.原因はステレオカメラのキャリブレーションの精度の低さ,アンカの取付誤差による推定誤差などが考えられる.

\section{プロペラの推力測定}
田中くんにご協力の下2種類のプロペラの推力測定を行った.1種類はもう片方の約1.5倍の推力が発生させることがわかった.飛行実験で用いたのは推力の弱い方であったため,以前に述べたヨー角の回転や制御の不安定さが発生したと考えられる.あとは4つのモータの推力も計測し,パワーに差がないことも確認したい.









\section{今後の予定}

\begin{itemize}
\itemsep=-1ex
  \item 修論書く
\end{itemize}

%\begin{thebibliography}{99}
%\bibitem{1}Pixracer, https://docs.px4.io/v1.9.0/en/flight\_controller/pixracer.html
%\bibitem{2}Servo Gripper, http://ardupilot.org/copter/docs/common-gripper-servo.html
%\bibitem{3}Electro Permanent Magnet Gripper (EPM688), http://ardupilot.org/copter/docs/common-electro-permanent-magnet-gripper.html
%\bibitem{4}Nica Drone, https://nicadrone.com/products/epm-v3
%\end{thebibliography}





%\begin{thebibliography}{1}
%
%\small
%
%\vspace{-2mm}
%\bibitem{1}
%\label{1}
%Krzysztof Cisek,``Ultra-Wide Band Real Time Location Systems: Practical
%Implementation and UAV Performance Evaluation''
%
%%\bibitem{ラベル}
%%著者,
%%題名,
%%誌名+ページ,
%%年月.
%
%\small
%
%\vspace{-2mm}
%\bibitem{2}
%\label{2}
% M. Pelka, G. Goronzy, and H. Hellbr¨uck, “Iterative approach for
%anchor configuration of positioning systems,” ICT Express, vol. 2,
%no. 1, pp. 1–4, 2016.
%
%
%\small
%
%\vspace{-2mm}
%\bibitem{3}
%\label{3}
%A. Norrdine, “An algebraic solution to the multilateration problem,”
%in Proceedings of the 15th International Conference on Indoor Posi-
%tioning and Indoor Navigation, Sydney, Australia, vol. 1315, 2012.
%
%\end{thebibliography}



%式
%\begin{eqnarray}
%\label{}
%\end{eqnarray}

%\begin{equation}
%\label{}
%\end{equation}

%%箇条書き
%\begin{itemize}
%\itemsep=-1ex
%  \item 
%  \item 
%  \item 
%  \item 
%  \item 
%  \item 
%\end{itemize}

%%図
%\begin{figure}[htbp]
%  \begin{center}
%     \includegraphics[width=1\linewidth]{}
%     \caption{}
%     \label{}
%  \end{center}
%\end{figure}

%%表
%\begin{table}[htbp]
%  \begin{center}
%  \caption{}
%  \label{}
%  \begin{tabular}{|c||c|c|c|}	\hline
%  &&& \\ \hline
%  &&&\\
%  &&&\\ \hline
%  \end{tabular}
%  \end{center}
%\end{table}

%スペースを詰める,あける
%\vspace{-2zh}
%\vspace{2zh}

%  参考文献
%%%%%%%%%%%%%%%%%%%%%%%%%%%%%%%%%%%%%%%%%%%%%%%%%%%%%%%%%%%%%%%%%%%%%%%%%%


\end{document}
