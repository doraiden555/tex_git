%%%%%%%%%%%%%%%%%%%%%%%%%%%%%%%%%%%%%%%%%%%%%%%%%%%%%%%%%%%%%%%%%%%%%%%%%%%
%  ロボティクス研  研究報告用TEXファイル  前刷り (旧田中研フォーマットベース)
%  Resume.tex
%
%  2003.03.14	T.Koyama
%  2005.04.11	H.Ohtake
%  2016.03.24	Y.Higashi
%%%%%%%%%%%%%%%%%%%%%%%%%%%%%%%%%%%%%%%%%%%%%%%%%%%%%%%%%%%%%%%%%%%%%%%%%%%
\documentclass[a4paper]{jarticle}
\usepackage{Resume}
\usepackage[dvipdfmx]{color,graphicx}
\usepackage{slashbox}
\usepackage{amsmath}
\usepackage{textgreek}%ギリシャ文字を立てにするパッケージ
\usepackage{nidanfloat}%横長の図を1ページ内にうまく挿入する

%本文圧縮コマンド(本文参照)
%\renewcommand{\baselinestretch}{0.75}

\begin{document}
\twocolumn[
%%%%%%%%%%%%%%%%%%%%%%%%%%%%%%%%%%%%%%%%%%%%%%%%%%%%%%%%%%%%%%%%%%%%%%%%%%%
%  タイトル・氏名
%%%%%%%%%%%%%%%%%%%%%%%%%%%%%%%%%%%%%%%%%%%%%%%%%%%%%%%%%%%%%%%%%%%%%%%%%%%
\vspace*{10mm}
\begin{center}
	{\Large \gt 2019/11/26 飛翔ロボットミーティング} \\
\end{center}
\begin{flushright}
\begin{tabular}{c@{~}r}
機械設計学専攻	& ロボティクス研究室	\\
18623117		& 中村 翔太		\\
\end{tabular}
\end{flushright}
\vspace{1em}
]

%%%%%%%%%%%%%%%%%%%%%%%%%%%%%%%%%%%%%%%%%%%%%%%%%%%%%%%%%%%%%%%%%%%%%%%%%%%
%  本文
%%%%%%%%%%%%%%%%%%%%%%%%%%%%%%%%%%%%%%%%%%%%%%%%%%%%%%%%%%%%%%%%%%%%%%%%%%%
%%%%%%%%%%%%%%%%%%%%%%%%%%%%%%%%%%%%%%%%%%%%%%%%%%%%%%%%%%%%%%%%%%%%%%%%%%%
%%%%%%%%%%%%%%%%%%%%%%%%%%%%%%%%%%%%%%%%%%%%%%%%%%%%%%%%%%%%%%%%%%%%%%%%%%%

\section{UWB発散値フィルタリング機能の改善}
前回のミーティングにて述べたUWBからの値が発散した際のフィルタリング機能が働いていない問題を解決した.前に書いていたプログラムが上手く働かなかった理由は配列を別の配列に代入した際,コピーを行っていたつもりであったが,実は参照渡しを行ってしまっていたために,意図しない挙動が発生してしまっていことが原因である.発散した値を弾くためにDD\_old[x](xは0から3の対応するUWBアンカ)という変数に前に送られてきたUWBの距離データを一旦バッファし,次に送られてきた距離データ(DD[x])と比べた際に変化量がしきい値を上回った場合,その直近の値を破棄するというものである.このDD\_oldという値を初期化する際に,DD\_old[x] = DD[x]としてしまったために,DD\_oldとDDが同じ参照先(同ID)を持つようになり,常にDD\_oldとDDが同じ値を格納するようになってしまった.Pythonでは配列1=配列2という処理を書くと,配列2の値が配列1にコピーされるのではなく,同じ参照先(ID)を参照するようになるため,配列1,2のどちらかが更新されると,それに伴ってもう一つの配列の値も更新される.したがって,DD\_old = DDは値のコピーではなく,同じ値をとるような処理になっていることがわかる.そのため,DD\_oldとDDの変化量を取った際に,値が同じなため,必ず0を返すようになってしまっていた.対策としてはb = a[:]のようにスライスを使う.若しくはc = copy.deepcopy(a)というcopyモジュールのdeepcopyという機能を用いるのが適している.あとは単純にfor文で回して要素を一つづつコピーする方法もある.Pythonに限らずこの深いコピー,浅いコピーといったことはミスを犯しがちなので気をつける必要がある.この修正プログラムを用いて実験したところ,きちんと値を弾けるようになったので動作確認も完了している.

\section{オプティカルフローセンサから正確な値が得られない原因及び実験装置の改善}
前回の実験で,UWBとオプティカルフローセンサの値を統合した位置推定値の精度はそこそこだが,オプティカルフローセンサ値の単純積分値の精度が悪いという結果を得た(fig. 3).


この悪い精度の原因は梁がクアッド,センサに与える振動が原因だと考えていたが,実はオプティカルフローセンサのカメラの画角内に動かない障害物(台の足)がかぶっていたため,正確な流量を計測できていなかったことが本当の原因だということがわかった.次の図のようにクアッドを台に固定しているが,オプティカルフローセンサの画角を全く考慮できておらず,カメラの画角内に台の足がもろに入り込んでしまっていることがわかる.



\begin{figure}[htbp]
  \begin{center}
     \includegraphics[width=1\linewidth]{./figure/quad_with_opt.jpg}
     \caption{台に取り付けられたクアッド}
     \label{}
  \end{center}
\end{figure}

画角42°という仕様から高さ(800mm)を考慮して画像範囲を計算すると,カメラレンズから垂直におろした線から左右奥行方向に307mmの範囲は計測範囲内であることがわかった(fig. 2).実際にスケールで計測してみたがやはり,307mmの範囲に足が入り込んでいた.これを改善するため,クアッドをさらに前に突き出した位置で固定し,さらにアルミフレームを共締めして振動も拾わないようにした.そのようにさらに実験器具を改良した上で再度同様の実験を行ったところ,次のfig. 4の結果を得た.

\begin{figure}[htbp]
  \begin{center}
     \includegraphics[width=1\linewidth]{./figure/opt_obstacle.png}
     \caption{センサの画角と高さの関係}
     \label{}
  \end{center}
\end{figure}


\begin{figure}[htbp]
  \begin{center}
     \includegraphics[width=1\linewidth]{./figure/opt_sum.png}
     \caption{オプティカルフローセンサから得た速度の積分値}
     \label{}
  \end{center}
\end{figure}

\begin{figure}[htbp]
  \begin{center}
     \includegraphics[width=1\linewidth]{./figure/UWB_and_opt.png}
     \caption{画角を考慮し改良した上で実験した推定結果及び積分値}
     \label{}
  \end{center}
\end{figure}

この実験ではスペースが十分確保できなかったため,前回動かしたような正方形のトレースではなく,半分の二等辺三角形をトレースする形になっている.fig. 3に比べ,きちんと対角線の長さ(=1m)分の移動量を計測できていることが分かる.しかし,やはり今回も右下の角(10秒から17秒付近)にて大きな分散が生じていることが分かる.最初はマルチパスによる距離の発散の影響が疑われたが,フィルタリング機能が適切に働いていることから,これは原因として考えにくい.そこで本実験でのTagと各Anchorとの計測距離を次に示す.

\begin{figure}[htbp]
  \begin{center}
     \includegraphics[width=1\linewidth]{./figure/UWB_distance.png}
     \caption{TagとそれぞれAnchorとの距離}
     \label{}
  \end{center}
\end{figure}

dd1, dd3, dd4は定常状態において概ね定常値をとっているが,dd2は右下の角に当たる点に移動した際に値が大きくぶれていることが分かる.これは,飛翔部屋における障害物(椅子や机)によって電波が回折し,短い距離ながらマルチパスが発生したため,このような計測値のぶれを生んだと考えられる.次回は下の実験室にてLOS環境を確保した上で再度推定の実験を行いたい.また,分散やフィルタの調整,位置制御用のプログラムが書け次第(飛行制御プログラムに粗があるので修正する予定)順次飛行試験にも入っていくつもりである.




\section{今後の予定}

\begin{itemize}
\itemsep=-1ex
  \item LOS環境を構築しての推定実験
  \item 飛行制御のための準備(ハード,ソフト)
\end{itemize}

%\begin{thebibliography}{99}
%\bibitem{1}Pixracer, https://docs.px4.io/v1.9.0/en/flight\_controller/pixracer.html
%\bibitem{2}Servo Gripper, http://ardupilot.org/copter/docs/common-gripper-servo.html
%\bibitem{3}Electro Permanent Magnet Gripper (EPM688), http://ardupilot.org/copter/docs/common-electro-permanent-magnet-gripper.html
%\bibitem{4}Nica Drone, https://nicadrone.com/products/epm-v3
%\end{thebibliography}





%\begin{thebibliography}{1}
%
%\small
%
%\vspace{-2mm}
%\bibitem{1}
%\label{1}
%Krzysztof Cisek,``Ultra-Wide Band Real Time Location Systems: Practical
%Implementation and UAV Performance Evaluation''
%
%%\bibitem{ラベル}
%%著者,
%%題名,
%%誌名+ページ,
%%年月.
%
%\small
%
%\vspace{-2mm}
%\bibitem{2}
%\label{2}
% M. Pelka, G. Goronzy, and H. Hellbr¨uck, “Iterative approach for
%anchor configuration of positioning systems,” ICT Express, vol. 2,
%no. 1, pp. 1–4, 2016.
%
%
%\small
%
%\vspace{-2mm}
%\bibitem{3}
%\label{3}
%A. Norrdine, “An algebraic solution to the multilateration problem,”
%in Proceedings of the 15th International Conference on Indoor Posi-
%tioning and Indoor Navigation, Sydney, Australia, vol. 1315, 2012.
%
%\end{thebibliography}



%式
%\begin{eqnarray}
%\label{}
%\end{eqnarray}

%\begin{equation}
%\label{}
%\end{equation}

%%箇条書き
%\begin{itemize}
%\itemsep=-1ex
%  \item 
%  \item 
%  \item 
%  \item 
%  \item 
%  \item 
%\end{itemize}

%%図
%\begin{figure}[htbp]
%  \begin{center}
%     \includegraphics[width=1\linewidth]{}
%     \caption{}
%     \label{}
%  \end{center}
%\end{figure}

%%表
%\begin{table}[htbp]
%  \begin{center}
%  \caption{}
%  \label{}
%  \begin{tabular}{|c||c|c|c|}	\hline
%  &&& \\ \hline
%  &&&\\
%  &&&\\ \hline
%  \end{tabular}
%  \end{center}
%\end{table}

%スペースを詰める,あける
%\vspace{-2zh}
%\vspace{2zh}

%  参考文献
%%%%%%%%%%%%%%%%%%%%%%%%%%%%%%%%%%%%%%%%%%%%%%%%%%%%%%%%%%%%%%%%%%%%%%%%%%


\end{document}
