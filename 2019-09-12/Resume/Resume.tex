%%%%%%%%%%%%%%%%%%%%%%%%%%%%%%%%%%%%%%%%%%%%%%%%%%%%%%%%%%%%%%%%%%%%%%%%%%%
%  ロボティクス研  研究報告用TEXファイル  前刷り (旧田中研フォーマットベース)
%  Resume.tex
%
%  2003.03.14	T.Koyama
%  2005.04.11	H.Ohtake
%  2016.03.24	Y.Higashi
%%%%%%%%%%%%%%%%%%%%%%%%%%%%%%%%%%%%%%%%%%%%%%%%%%%%%%%%%%%%%%%%%%%%%%%%%%%
\documentclass[a4paper]{jarticle}
\usepackage{Resume}
\usepackage[dvipdfmx]{color,graphicx}

%本文圧縮コマンド(本文参照)
%\renewcommand{\baselinestretch}{0.75}


\begin{document}
\twocolumn[
%%%%%%%%%%%%%%%%%%%%%%%%%%%%%%%%%%%%%%%%%%%%%%%%%%%%%%%%%%%%%%%%%%%%%%%%%%%
%  タイトル・氏名
%%%%%%%%%%%%%%%%%%%%%%%%%%%%%%%%%%%%%%%%%%%%%%%%%%%%%%%%%%%%%%%%%%%%%%%%%%%
\vspace*{10mm}
\begin{center}
	{\Large \gt 20XX/4/1 何とか無事に卒業する方法に関する研究} \\%書き換え必要
\end{center}
\begin{flushright}
\begin{tabular}{c@{~}r}
機械工学課程	& ロボティクス研究室	\\%書き換え必要
00123456		& 工繊 ろぼ太郎		\\%書き換え必要
\end{tabular}
\end{flushright}
\vspace{1em}
]

%%%%%%%%%%%%%%%%%%%%%%%%%%%%%%%%%%%%%%%%%%%%%%%%%%%%%%%%%%%%%%%%%%%%%%%%%%%
%  本文
%%%%%%%%%%%%%%%%%%%%%%%%%%%%%%%%%%%%%%%%%%%%%%%%%%%%%%%%%%%%%%%%%%%%%%%%%%%

%%%%%%%%%%%%%%%%%%%%%%%%%%%%%%%%%%%%%%%%%%%%%%%%%%%%%%%%%%%%%%%%%%%%%%%%%%%
\section{緒論}
%%%%%%%%%%%%%%%%%%%%%%%%%%%%%%%%%%%%%%%%%%%%%%%%%%%%%%%%%%%%%%%%%%%%%%%%%%%
本研究は,急務とされている卒業のために有効な方法の1つとして,前刷りのスタイルを説明する.

%%%%%%%%%%%%%%%%%%%%%%%%%%%%%%%%%%%%%%%%%%%%%%%%%%%%%%%%%%%%%%%%%%%%%%%%%%%
\section{スタイル}
%%%%%%%%%%%%%%%%%%%%%%%%%%%%%%%%%%%%%%%%%%%%%%%%%%%%%%%%%%%%%%%%%%%%%%%%%%%
基本的にこのまま使えるはずです.
本文が多すぎて規定のページ数にどうしても収まらない人は,このファイルの13行目,
\begin{verbatim}
%\renewcommand{\baselinestretch}{0.75}
\end{verbatim}
の文頭の``\%''を削除して下さい.

%%%%%%%%%%%%%%%%%%%%%%%%%%%%%%%%%%%%%%%%%%%%%%%%%%%%%%%%%%%%%%%%%%%%%%%%%%%
\section{図の入れ方}
%%%%%%%%%%%%%%%%%%%%%%%%%%%%%%%%%%%%%%%%%%%%%%%%%%%%%%%%%%%%%%%%%%%%%%%%%%%
ソースを参照して下さい.
\begin{figure}[htbp]
  \begin{center}
     \includegraphics[width=0.4\linewidth]{Fig.eps}
     %\resizebox{40mm}{!}{\includegraphics{Fig.eps}}
     \caption{Figure test}
     \label{fig:FigTest}
  \end{center}
\end{figure}

%%%%%%%%%%%%%%%%%%%%%%%%%%%%%%%%%%%%%%%%%%%%%%%%%%%%%%%%%%%%%%%%%%%%%%%%%%%
\section{表の入れ方}
%%%%%%%%%%%%%%%%%%%%%%%%%%%%%%%%%%%%%%%%%%%%%%%%%%%%%%%%%%%%%%%%%%%%%%%%%%%
ソースを参照して下さい.
\begin{table}[htbp]
  \begin{center}
  \caption{Table test}
  \label{tab:TabTest}
  \begin{tabular}{|c||c|c|c|c|c|}				\hline
  あ行	& あ	& い	& う	& え	& お		\\\hline
  か行	& か	& き	& く	& け	& こ		\\\hline
  \end{tabular}
  \end{center}
\end{table}

%%%%%%%%%%%%%%%%%%%%%%%%%%%%%%%%%%%%%%%%%%%%%%%%%%%%%%%%%%%%%%%%%%%%%%%%%%%
\section{ラベル(label)のすすめ}
%%%%%%%%%%%%%%%%%%%%%%%%%%%%%%%%%%%%%%%%%%%%%%%%%%%%%%%%%%%%%%%%%%%%%%%%%%%
ソースを見れば分かりますが,図表や式にラベルを貼る事をお薦めします.
これは,Fig.\ref{fig:FigTest}
と使うことで,図などの番号を自動的につけてくれます.
急遽,図の変更があったときなどに便利ですので,常につけておきましょう.

ただし,新しくラベルを貼って最初のtex処理では,処理が2回行われます.

%%%%%%%%%%%%%%%%%%%%%%%%%%%%%%%%%%%%%%%%%%%%%%%%%%%%%%%%%%%%%%%%%%%%%%%%%%%
\section{箇条書きの入れ方}
%%%%%%%%%%%%%%%%%%%%%%%%%%%%%%%%%%%%%%%%%%%%%%%%%%%%%%%%%%%%%%%%%%%%%%%%%%%
行間が空きすぎるので,下のおまじないを使います.
\begin{verbatim}
\itemsep=-1ex
\end{verbatim}
出力結果はこんな感じ.

使用前
\begin{itemize}
  \item 学部(B)はビギナー.
  \item 修士(M)は召使い.
  \item 博士(D)は奴隷.
\end{itemize}

使用後
\begin{itemize}
\itemsep=-1ex
  \item 学部(B)はビギナー.
  \item 修士(M)は召使い.
  \item 博士(D)は奴隷.
\end{itemize}

%%%%%%%%%%%%%%%%%%%%%%%%%%%%%%%%%%%%%%%%%%%%%%%%%%%%%%%%%%%%%%%%%%%%%%%%%%%
\section{体裁の微調整}
%%%%%%%%%%%%%%%%%%%%%%%%%%%%%%%%%%%%%%%%%%%%%%%%%%%%%%%%%%%%%%%%%%%%%%%%%%%
\verb|\vspace{長さ} や \hspace{長さ} |を使うと縦方向,横方向の間隔の微調整ができます.
次章との間は\verb|\vspace{5mm}|で5mm分広くしてあります.

\vspace{5mm}
%%%%%%%%%%%%%%%%%%%%%%%%%%%%%%%%%%%%%%%%%%%%%%%%%%%%%%%%%%%%%%%%%%%%%%%%%%%
\section{結論}
%%%%%%%%%%%%%%%%%%%%%%%%%%%%%%%%%%%%%%%%%%%%%%%%%%%%%%%%%%%%%%%%%%%%%%%%%%%
頑張るしかない.

%%%%%%%%%%%%%%%%%%%%%%%%%%%%%%%%%%%%%%%%%%%%%%%%%%%%%%%%%%%%%%%%%%%%%%%%%%%
%  参考文献
%%%%%%%%%%%%%%%%%%%%%%%%%%%%%%%%%%%%%%%%%%%%%%%%%%%%%%%%%%%%%%%%%%%%%%%%%%%
\begin{thebibliography}{1}

\small
\vspace{-2mm}
\bibitem{HaroJSME1st}
\label{HaroJSME1st}
小山猛, 山藤和男, 田中孝之, ``介護用装着型ヒューマン・アシスト装置に関する研究(第1報,コンセプト,システム設計と実機の開発)'', 日本機械学会論文集 C編, Vol.66, No.651, 155-160 (2000).

\vspace{-2mm}
\bibitem{Icma2000}
\label{Icma2000}
T. Koyama, M. Q. Feng and T. Tanaka, ``Development and Motion Control of a Wearable Human Assisting Robot for Nursing Use'', Proceedings of the International Conference on Machine Automation, 555-560 (2000).

\vspace{-2mm}
\bibitem{tex_comand}
\label{tex_comand}
``LaTeXコマンドシート一覧'', http://www002.upp.so-net.ne.jp/latex/index.html, 2016年3月24日閲覧.

%\bibitem{ラベル}
%著者,
%題名,
%誌名+ページ,
%年月.

\end{thebibliography}

%%%絵の入れ方%%%%%%%%%%%%%%%%%%%%%%%%%%%%%%%%%%%%%%%%%%%%%%%%%%%%%%%%%%%%%%
%\begin{figure}[htbp]
%  \begin{center}
%    \resizebox{70mm}{!}{\includegraphics{xxx.eps}}
%    \caption{Diagram of HFS to HARO}
%    \label{fig:HFS2HARO}
%  \end{center}
%\end{figure}

\end{document}
