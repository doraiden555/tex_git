%%%%%%%%%%%%%%%%%%%%%%%%%%%%%%%%%%%%%%%%%%%%%%%%%%%%%%%%%%%%%%%%%%%%%%%%%%%
%  ロボティクス研  研究報告用TEXファイル  前刷り (旧田中研フォーマットベース)
%  Resume.tex
%
%  2003.03.14	T.Koyama
%  2005.04.11	H.Ohtake
%  2016.03.24	Y.Higashi
%%%%%%%%%%%%%%%%%%%%%%%%%%%%%%%%%%%%%%%%%%%%%%%%%%%%%%%%%%%%%%%%%%%%%%%%%%%
\documentclass[a4paper]{jarticle}
\usepackage{Resume}
\usepackage[dvipdfmx]{color,graphicx}
\usepackage{slashbox}
\usepackage{amsmath}
\usepackage{textgreek}%ギリシャ文字を立てにするパッケージ
\usepackage{nidanfloat}%横長の図を1ページ内にうまく挿入する

%本文圧縮コマンド(本文参照)
%\renewcommand{\baselinestretch}{0.75}

\begin{document}
\twocolumn[
%%%%%%%%%%%%%%%%%%%%%%%%%%%%%%%%%%%%%%%%%%%%%%%%%%%%%%%%%%%%%%%%%%%%%%%%%%%
%  タイトル・氏名
%%%%%%%%%%%%%%%%%%%%%%%%%%%%%%%%%%%%%%%%%%%%%%%%%%%%%%%%%%%%%%%%%%%%%%%%%%%
\vspace*{10mm}
\begin{center}
	{\Large \gt 2019/12/18 飛翔ロボットミーティング} \\
\end{center}
\begin{flushright}
\begin{tabular}{c@{~}r}
機械設計学専攻	& ロボティクス研究室	\\
18623117		& 中村 翔太		\\
\end{tabular}
\end{flushright}
\vspace{1em}
]

%%%%%%%%%%%%%%%%%%%%%%%%%%%%%%%%%%%%%%%%%%%%%%%%%%%%%%%%%%%%%%%%%%%%%%%%%%%
%  本文
%%%%%%%%%%%%%%%%%%%%%%%%%%%%%%%%%%%%%%%%%%%%%%%%%%%%%%%%%%%%%%%%%%%%%%%%%%%
%%%%%%%%%%%%%%%%%%%%%%%%%%%%%%%%%%%%%%%%%%%%%%%%%%%%%%%%%%%%%%%%%%%%%%%%%%%
%%%%%%%%%%%%%%%%%%%%%%%%%%%%%%%%%%%%%%%%%%%%%%%%%%%%%%%%%%%%%%%%%%%%%%%%%%%

\section{UWB+オプティカルフローセンサの構成で位置制御}
以前より推定アルゴリズム,位置制御アルゴリズムを構築していたUWB+オプティカルフローセンサを用いた構成にてドローンの位置を制御する実験を行った,内容は以前にも行った原点でのホバリングである.その結果をfig. 1に示す.原点からの距離の標準偏差は0.113 m,最大誤差は0.625 mとなった.この数字は以前にUWBのみで飛行させた時(0.083 m,0.406 m)よりも悪くなってしまっている.理由は今の所わからないが,以前の飛行時間の1分に比べ今回は4分と長時間飛ばしたことやオプティカルフローセンサが搭載されたことで,今までは時間分解能が粗く,計測できていなかった移動距離まで計測できるようになったためなどが理由として考えている.


体感としては,以前(オプティカルフローセンサ未搭載)はUWBの値が飛行中に揺らぐため,ある程度は位置をとどめてくれるが,ふらついた飛行であった.しかし,今回の実験ではそういったことが少なくなり,比較的安定したホバリングであったと感じる.カルマンフィルタの分散をもう少し調整し,観測値をオプティカルフローセンサ側に寄せても良い感じがする.

\begin{figure}[htbp]
  \begin{center}
     \includegraphics[width=1\linewidth]{./figure/with_opt.png}
     \caption{オプティカルフローセンサの値も用いた場合の原点ホバリング結果}
     \label{}
  \end{center}
\end{figure}





\section{ステレオカメラの件}
前回,ステレオカメラを同期させるには同期用のケーブルが必要であると述べたが,そのケーブルの存在を確認した.それを用いて動画(静止画)撮影の同期を取れるようアプローチしてみた.手法としては,どちらかのカメラをプライマリカメラとして設定し,もう片方をセカンダリカメラとして設定する.そして,セカンダリカメラは同期モードにし,プライマリカメラから同期用の割り込み信号を出し,それをセカンダリカメラが拾う形で2台間の同期をとることが出来る.実際に行ったところ,確かに開始のタイミングは完全に合わせることが可能であったが,2代とも100fpsで設定したにも関わらず,2台間のフレームレートが全く合わず,プライマリカメラは100fpsで撮影できているが,セカンダリカメラは60fpsでしか撮影できないという結果を得た.恐らくCPUの処理の限界の問題だと考える.

次に2台間の同期は初めから諦め,静止画を100fpsで撮影できるようアプローチした.以前までは100fpsで撮影した動画をPythonの機能で分割することで一秒間に100枚の静止画を得ていた.しかし,この静止画にする作業に非常に時間を要していた(1分の動画で数10分)ため,100fpsで静止画を撮影できるようにした.初め,両カメラで100fpsで撮影したところ,スキップされるフレームがとても多く(7フレームに1まいくらいくらい),とても100fpsでは撮影できなかった.しかし,これの原因は画像の解像度を高くしてしまっていたためであり,以前の画質を落とした設定に戻せば問題なくフレーム落ちすることなく,100fpsにてカメラ2台で画像を撮影することができた.設定が変わってしまっていた原因はカメラ同期をするために設定をいじっている際に画質の設定まで変わってしまったためである.また,飛翔部屋のPCのWindowsにPython環境を構築したため,今までは別のPCの環境で行っていた動画の分割や位置座標を求める計算などを全て1つのPC内で完結できるようになった.また,画像の記録をHDDからSSDへと保存先を改めたことでも大幅な作業効率アップができた.

ここで先の同期の話に戻るが,2台のカメラで100fpsで問題なく撮影できたため,画質の設定を見直せば,きちんと同期が取れた状態で100fpsで撮影が可能かもしれない.同期が取れていなくてもPythonのプログラム側で同期が取れるよう組んであるため,これ以上の検証は修論が終わってからに持ち越したいと思う.


\section{SIIのスライド作り}
来週も発表練習があるため,今週ご指摘頂いた点をきちんと修正した後臨みたいと思う.


\section{今後の予定}

\begin{itemize}
\itemsep=-1ex
  \item 真値をロギングした状態(ステレオカメラを用いて)で以前の制御アルゴリズム,開発した制御アルゴリズムで飛行実験
  \item 修論のためのデータ集め,まとめ
\end{itemize}

%\begin{thebibliography}{99}
%\bibitem{1}Pixracer, https://docs.px4.io/v1.9.0/en/flight\_controller/pixracer.html
%\bibitem{2}Servo Gripper, http://ardupilot.org/copter/docs/common-gripper-servo.html
%\bibitem{3}Electro Permanent Magnet Gripper (EPM688), http://ardupilot.org/copter/docs/common-electro-permanent-magnet-gripper.html
%\bibitem{4}Nica Drone, https://nicadrone.com/products/epm-v3
%\end{thebibliography}





%\begin{thebibliography}{1}
%
%\small
%
%\vspace{-2mm}
%\bibitem{1}
%\label{1}
%Krzysztof Cisek,``Ultra-Wide Band Real Time Location Systems: Practical
%Implementation and UAV Performance Evaluation''
%
%%\bibitem{ラベル}
%%著者,
%%題名,
%%誌名+ページ,
%%年月.
%
%\small
%
%\vspace{-2mm}
%\bibitem{2}
%\label{2}
% M. Pelka, G. Goronzy, and H. Hellbr¨uck, “Iterative approach for
%anchor configuration of positioning systems,” ICT Express, vol. 2,
%no. 1, pp. 1–4, 2016.
%
%
%\small
%
%\vspace{-2mm}
%\bibitem{3}
%\label{3}
%A. Norrdine, “An algebraic solution to the multilateration problem,”
%in Proceedings of the 15th International Conference on Indoor Posi-
%tioning and Indoor Navigation, Sydney, Australia, vol. 1315, 2012.
%
%\end{thebibliography}



%式
%\begin{eqnarray}
%\label{}
%\end{eqnarray}

%\begin{equation}
%\label{}
%\end{equation}

%%箇条書き
%\begin{itemize}
%\itemsep=-1ex
%  \item 
%  \item 
%  \item 
%  \item 
%  \item 
%  \item 
%\end{itemize}

%%図
%\begin{figure}[htbp]
%  \begin{center}
%     \includegraphics[width=1\linewidth]{}
%     \caption{}
%     \label{}
%  \end{center}
%\end{figure}

%%表
%\begin{table}[htbp]
%  \begin{center}
%  \caption{}
%  \label{}
%  \begin{tabular}{|c||c|c|c|}	\hline
%  &&& \\ \hline
%  &&&\\
%  &&&\\ \hline
%  \end{tabular}
%  \end{center}
%\end{table}

%スペースを詰める,あける
%\vspace{-2zh}
%\vspace{2zh}

%  参考文献
%%%%%%%%%%%%%%%%%%%%%%%%%%%%%%%%%%%%%%%%%%%%%%%%%%%%%%%%%%%%%%%%%%%%%%%%%%


\end{document}
