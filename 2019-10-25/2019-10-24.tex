%%%%%%%%%%%%%%%%%%%%%%%%%%%%%%%%%%%%%%%%%%%%%%%%%%%%%%%%%%%%%%%%%%%%%%%%%%%
%  ロボティクス研  研究報告用TEXファイル  前刷り (旧田中研フォーマットベース)
%  Resume.tex
%
%  2003.03.14	T.Koyama
%  2005.04.11	H.Ohtake
%  2016.03.24	Y.Higashi
%%%%%%%%%%%%%%%%%%%%%%%%%%%%%%%%%%%%%%%%%%%%%%%%%%%%%%%%%%%%%%%%%%%%%%%%%%%
\documentclass[a4paper]{jarticle}
\usepackage{Resume}
\usepackage[dvipdfmx]{color,graphicx}
\usepackage{slashbox}
\usepackage{amsmath}
\usepackage{textgreek}%ギリシャ文字を立てにするパッケージ
\usepackage{nidanfloat}%横長の図を1ページ内にうまく挿入する

%本文圧縮コマンド(本文参照)
%\renewcommand{\baselinestretch}{0.75}

\begin{document}
\twocolumn[
%%%%%%%%%%%%%%%%%%%%%%%%%%%%%%%%%%%%%%%%%%%%%%%%%%%%%%%%%%%%%%%%%%%%%%%%%%%
%  タイトル・氏名
%%%%%%%%%%%%%%%%%%%%%%%%%%%%%%%%%%%%%%%%%%%%%%%%%%%%%%%%%%%%%%%%%%%%%%%%%%%
\vspace*{10mm}
\begin{center}
	{\Large \gt 2019/10/24 飛翔ロボットミーティング} \\
\end{center}
\begin{flushright}
\begin{tabular}{c@{~}r}
機械設計学専攻	& ロボティクス研究室	\\
18623117		& 中村 翔太		\\
\end{tabular}
\end{flushright}
\vspace{1em}
]

%%%%%%%%%%%%%%%%%%%%%%%%%%%%%%%%%%%%%%%%%%%%%%%%%%%%%%%%%%%%%%%%%%%%%%%%%%%
%  本文
%%%%%%%%%%%%%%%%%%%%%%%%%%%%%%%%%%%%%%%%%%%%%%%%%%%%%%%%%%%%%%%%%%%%%%%%%%%
%%%%%%%%%%%%%%%%%%%%%%%%%%%%%%%%%%%%%%%%%%%%%%%%%%%%%%%%%%%%%%%%%%%%%%%%%%%
%%%%%%%%%%%%%%%%%%%%%%%%%%%%%%%%%%%%%%%%%%%%%%%%%%%%%%%%%%%%%%%%%%%%%%%%%%%

\section{gitの勉強}
今まではGitはGithubにて人のリポジトリをクローンしてArduinoのライブラリやPythonのモジュールを使用させてもらうにとどまっていたが,現在書いているクアッドの位置推定プログラムなどのバージョン管理をしようということでGitを導入することにした.今までは開発中のプログラムの別バージョンを作りたい場合,名前を微妙に変えてバージョンを進めていたが,これだと,プログラムのどの箇所を変えたかや,どのバージョンが最新かが分からなくなることがあった.そこでGitを使い,こまめにバージョン管理をすることによってこれらの問題は解決することができるようになった.また,クアッドに搭載したRspberry Pi上のプログラムをいじる際にはいちいちRaspberry PiにSSH接続し,対象のプログラムを書き換える,もしくはRaspberry Piに接続せずに予め書いたプログラムをSSH接続した際にコピーするという面倒な作業を行っていた.しかし,Gitを用いることで,予め書いたプログラムをリモートリポジトリに上げることでリポジトリごとRaspberry PiにPullしてくることが可能なため,開発を早くすることができた.開発においてかなり労力を削減可能なので是非ロボティクスのQiitaを参考に導入してみてほしい.あと将来的に修士の研究はプログラムや論文を含め,全てGitにて引き継ぐことを予定しているため,現在進行系で引き継ぎ用のリポジトリやREADMEを作成中である.


\section{新しいクアッドへの装備の換装}
以前から製作していた新しいクアッドに対して装備の換装を行った.まず一旦Navio2を積まずにPixracerのみを積んで組み上げた(fig. 1).そして実験室にて飛行試験を行ったところ,設計通り10分弱ほど飛行させることが可能であった.また,この時の重量はバッテリー含め872.5gであった.次にfig. 2の様に以前のクアッドと変わらない装備にて飛行もさせたが,以前ほどスラストに余裕がない感じではなく,安定して離陸させることが可能であった.この状態の重量は1042gであった.飛行時間はまだ検証できていない.以前にも述べた通り,UARTを2つ読めるマイコンが必要なため,Teensyを乗せる必要がったが,手近にあったTeensyが3.6しかなかったため,これを積んでいるが,軽量化のためにTeensyLCなどに換装することを予定している.

\begin{figure}[htbp]
  \begin{center}
     \includegraphics[width=1\linewidth]{./figure/f330_pix.png}
     \caption{Pixracerのみ積んだクアッド}
     \label{}
  \end{center}
\end{figure}



\begin{figure}[htbp]
  \begin{center}
     \includegraphics[width=1\linewidth]{./figure/f330_navio2.png}
     \caption{以前のクアッドより必要なパーツを換装した状態のクアッド}
     \label{}
  \end{center}
\end{figure}

また,このプロペラは穴を拡げる追加工を行ったものであるが,飛行に関しては全く問題はないものの,柔らかいプロペラ特有の回転中にたわんで音が鳴ることやスラストが少し出し切れていない感じがするため,もう少し剛性の高いプロペラに替えれば,さらにモータの出力を効率よく引き出すことが可能であると考えている.


\section{オプティカルフローセンサの値も参照した位置推定}
以前にも示したfig. 3のシステム構成での動作検証及び,前章のような換装が済んだため,実際にクアッドを地面の上で移動させながら位置推定をする実験を行った.

\begin{figure}[htbp]
  \begin{center}
     \includegraphics[width=1\linewidth]{./figure/system.png}
     \caption{UWB及びオプティカルフローセンサにおけるシステムの構成図}
     \label{}
  \end{center}
\end{figure}

実験方法としては以前に行っていた方法と同様に地面に原点を取り,その位置から前後左右方向に1mずつ移動した点にマークを打ち,原点を含む5点を順に移動させて位置を推定するというものである.



\section{今後の予定}
\begin{itemize}
\itemsep=-1ex
  \item カルマンゲインの調整
 \item 位置推定値の精度検証(飛行中など)
\end{itemize}



%\begin{thebibliography}{99}
%\bibitem{1}Pixracer, https://docs.px4.io/v1.9.0/en/flight\_controller/pixracer.html
%\bibitem{2}Servo Gripper, http://ardupilot.org/copter/docs/common-gripper-servo.html
%\bibitem{3}Electro Permanent Magnet Gripper (EPM688), http://ardupilot.org/copter/docs/common-electro-permanent-magnet-gripper.html
%\bibitem{4}Nica Drone, https://nicadrone.com/products/epm-v3
%\end{thebibliography}





%\begin{thebibliography}{1}
%
%\small
%
%\vspace{-2mm}
%\bibitem{1}
%\label{1}
%Krzysztof Cisek,``Ultra-Wide Band Real Time Location Systems: Practical
%Implementation and UAV Performance Evaluation''
%
%%\bibitem{ラベル}
%%著者,
%%題名,
%%誌名+ページ,
%%年月.
%
%\small
%
%\vspace{-2mm}
%\bibitem{2}
%\label{2}
% M. Pelka, G. Goronzy, and H. Hellbr¨uck, “Iterative approach for
%anchor configuration of positioning systems,” ICT Express, vol. 2,
%no. 1, pp. 1–4, 2016.
%
%
%\small
%
%\vspace{-2mm}
%\bibitem{3}
%\label{3}
%A. Norrdine, “An algebraic solution to the multilateration problem,”
%in Proceedings of the 15th International Conference on Indoor Posi-
%tioning and Indoor Navigation, Sydney, Australia, vol. 1315, 2012.
%
%\end{thebibliography}



%式
%\begin{eqnarray}
%\label{}
%\end{eqnarray}

%\begin{equation}
%\label{}
%\end{equation}

%%箇条書き
%\begin{itemize}
%\itemsep=-1ex
%  \item 
%  \item 
%  \item 
%  \item 
%  \item 
%  \item 
%\end{itemize}

%%図
%\begin{figure}[htbp]
%  \begin{center}
%     \includegraphics[width=1\linewidth]{}
%     \caption{}
%     \label{}
%  \end{center}
%\end{figure}

%%表
%\begin{table}[htbp]
%  \begin{center}
%  \caption{}
%  \label{}
%  \begin{tabular}{|c||c|c|c|}	\hline
%  &&& \\ \hline
%  &&&\\
%  &&&\\ \hline
%  \end{tabular}
%  \end{center}
%\end{table}

%スペースを詰める,あける
%\vspace{-2zh}
%\vspace{2zh}

%  参考文献
%%%%%%%%%%%%%%%%%%%%%%%%%%%%%%%%%%%%%%%%%%%%%%%%%%%%%%%%%%%%%%%%%%%%%%%%%%


\end{document}
